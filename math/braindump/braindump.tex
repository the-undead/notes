\documentclass{article}

\usepackage[utf8]{inputenc}
\usepackage{amsmath, amssymb}
\usepackage{hyperref}
\usepackage{cancel}
\usepackage{tikz}
\usepackage{setspace}  
\usepackage[a4paper, margin=2cm]{geometry}  
\usepackage{graphicx}
\graphicspath{ {./images/} }
\newcommand{\img}[2][]{\includegraphics[width=0.5\textwidth,#1]{#2}}

\onehalfspacing 

\setlength{\parskip}{0pt}
\setlength{\parindent}{0pt}

\setlength{\baselineskip}{0pt}
\hypersetup{
    colorlinks=true,
    linkcolor=black,
    filecolor=black,
    urlcolor=black,
}

\title{Braindump}
\author{by me}
\date{\today}

\begin{document}

\maketitle

Warning: These are completely random notes, written in a probably unhelpful way.

\tableofcontents

\section{Triangle inequality theorem}

The triangle inequality theorem states:

The lenght of any side must be smaller than the sum of the other two sides.

\section{Divisibility by 4}

A number is divisible by 4 if the last two digits form a number that is divisible by 4.

\section{Find an equation that passes through two points}

Linear Function Equation:
$y=mx+b$

where $m$ is the slope and $b$ is the y-intercept

$m=\frac{y_2-y_1}{x_2-x_1}$


Basically just put the points in the equation.

TODO:fix section

\section{Fundamental geometric objects}

\subsection{Points}
Point: Exact location in space, has no size (no length,widht, depth), only position.
A point is indicated with a dot usually labled with a capital letter (P,Q,S,..)

\img{points.png}

A line is a straight object that is infinitly long and has no width. Like an infinite collection of point, going to infinity in both directions.

\img{line1.png}

\subsection{Lines}

This line that passes through the points P and Q is written like so:

$ \overleftrightarrow{PQ} $

Lines can also be denoted like so with lowercase letters:

\img{line2.png}

\subsection{Rays}
These are similar to a line but they have a starting point and only extend infinitely in one direction.

Example:

\img{ray1}

A ray that starts at point $P$ and passes through a point $Q$ is expressed as 

$\overrightarrow{PQ}$


The arrows tail in the expression is above the starting point.


\section{Segment}

\img{segment1}

Like a line but with start and end points, it does not extend infinitely.

A segment with Endpoints P and Q is denoted as $\overline{PQ}$

\end{document}